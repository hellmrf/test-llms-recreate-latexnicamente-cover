% arara: pdflatex
% Gerado por Gemini 3 Thinking
\documentclass[aspectratio=169, 12pt]{beamer}

% Pacotes essenciais
\usepackage[utf8]{inputenc}
\usepackage[T1]{fontenc}
\usepackage[brazil]{babel}
\usepackage{lmodern}     % Fontes vetoriais de alta qualidade
\usepackage{tikz}        % Para posicionamento e desenho da bandeira
\usepackage{hyperref}    % Para o link clicável
\usepackage{xcolor}      % Cores personalizadas

% Configurações visuais
\setbeamertemplate{navigation symbols}{} % Remove ícones de navegação
\setbeamercolor{background canvas}{bg=white} % Fundo branco

% Definição de cores baseadas na imagem
\definecolor{fadegray}{gray}{0.80} % Cinza claro para o texto do topo
\definecolor{codeblack}{gray}{0.2} % Cinza escuro para o comando
\definecolor{linkblue}{RGB}{50, 50, 180} % Azul/Roxo para o link
\definecolor{flaggreen}{RGB}{0, 156, 59}
\definecolor{flagyellow}{RGB}{255, 223, 0}
\definecolor{flagblue}{RGB}{0, 39, 118}

\begin{document}

\begin{frame}[plain]
\begin{tikzpicture}[remember picture, overlay]

    % 1. Texto do Topo (Faded)
    % Fonte Serifada, cor clara
    \node[anchor=north, yshift=-0.8cm] at (current page.north) {
        \parbox{0.9\paperwidth}{
            \centering % Ou justify, mas na imagem parece centralizado/blocado
            \rmfamily\color{fadegray}\small
            LAMPORT, Leslie. \LaTeX: A Document Preparation System. 2. ed. Reading:\\
            Addison-Wesley, 1994. ISBN 9780201529838.
        }
    };

    % 2. Título Principal
    % Fonte Sans-Serif, Negrito, Grande
    \node[anchor=center, yshift=1.5cm] at (current page.center) {
        \sffamily\bfseries\fontsize{30}{36}\selectfont Referências Bibliográficas em
    };

    % 3. O Comando em destaque
    % Fonte Typewriter, Grande
    \node[anchor=center, yshift=-0.2cm] at (current page.center) {
        \ttfamily\fontsize{45}{50}\selectfont
        \color{gray}\textbackslash cite\{\textbf{\color{black}ABNT}\}
    };

    % 4. A Referência Bibliográfica (Corpo)
    % Fonte Serifada, alinhada à esquerda visualmente
    \node[anchor=center, yshift=-2.5cm] at (current page.center) {
        \parbox{0.85\paperwidth}{
            \rmfamily\normalsize\color{gray}
            MARQUES, Daniel Ballester. \textbf{\color{black}Bib\LaTeX-ABNT}: Estilo para Bib\LaTeX\ compatível com
            as normas da ABNT. [S. l.], 2018. Disponível em:\\
            \href{https://ctan.org/pkg/biblatex-abnt}{\color{linkblue}\ttfamily https://ctan.org/pkg/biblatex-abnt}. \rmfamily Acesso em: 6 maio 2025.
        }
    };

    % 5. Rodapé: Bandeira do Brasil (Desenhada em TikZ para não precisar de imagem)
    \node[anchor=south west, xshift=1cm, yshift=0.8cm] at (current page.south west) {
        \begin{tikzpicture}[scale=0.04]
            \fill[flaggreen] (0,0) rectangle (15,10.5); % Retângulo
            \fill[flagyellow] (7.5, 9.3) -- (13.8, 5.25) -- (7.5, 1.2) -- (1.2, 5.25) -- cycle; % Losango
            \fill[flagblue] (7.5, 5.25) circle (2.6); % Círculo
            \draw[white, line width=2pt] (5, 5.8) to[out=-20, in=160] (10, 4.8); % Faixa simples
        \end{tikzpicture}
    };

    % 6. Rodapé: Logo LaTeXnicamente
    \node[anchor=south, yshift=0.8cm] at (current page.south) {
        \fontsize{20}{24}\selectfont
        \rmfamily \LaTeX nicamente
    };

\end{tikzpicture}
\end{frame}

\end{document}
