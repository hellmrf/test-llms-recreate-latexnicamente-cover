% arara: pdflatex
% Gerado por Perplexity Pro (Claude 4.5 Sonnet Thinking)
\documentclass{article}

% Página em proporção 16:9
\usepackage[
  paperwidth=16cm,
  paperheight=9cm,
  margin=0.8cm
]{geometry}

\usepackage[T1]{fontenc}
\usepackage[utf8]{inputenc}
\usepackage[brazil]{babel}
\usepackage{lmodern}
\usepackage{microtype}

\usepackage{tikz}
\usetikzlibrary{calc}

\pagestyle{empty}

\begin{document}

% Elementos sobrepostos (citação esmaecida, bandeira e logotipo)
\begin{tikzpicture}[remember picture,overlay]

  % Citação de Lamport ao fundo
  \node[anchor=north west,text opacity=0.25]
    at ([xshift=0pt,yshift=-8pt]current page.north west)
    {\footnotesize
      LAMPORT, Leslie. \LaTeX: A Document Preparation System.
      2.\ ed. Reading: Addison\hyp{}Wesley, 1994. ISBN 9780201529838.};

  % Bandeira do Brasil no canto inferior esquerdo
  \node[anchor=south west]
    at ([xshift=6pt,yshift=6pt]current page.south west){
      \begin{tikzpicture}[scale=0.09]
        \fill[green!60!black] (0,0) rectangle (20,14);
        \fill[yellow] (10,7) ++(0,7) -- ++(10,-7) -- ++(-10,-7)
                             -- ++(-10,7) -- cycle;
        \fill[blue] (10,7) circle (4);
        \draw[white,line width=0.7pt]
          (3,9) .. controls (7,11) and (13,3) .. (17,5);
      \end{tikzpicture}
    };

  % Logotipo "LaTeXnicamente" no rodapé central
  \node[anchor=south]
    at ([yshift=8pt]current page.south)
    {\Large \LaTeX nicamente};

\end{tikzpicture}

% Título central
\vspace*{1.8cm}
\begin{center}
  {\LARGE Referências Bibliográficas em\par}
  \vspace{0.9em}
  {\Huge\ttfamily \textbackslash cite\{ABNT\}\par}
\end{center}

\vfill

% Referência em estilo ABNT
\noindent\small
MARQUES, Daniel Ballester. \textbf{Bib\LaTeX-ABNT}: Estilo para Bib\LaTeX{}
compatível com as normas da ABNT. [S. l.], 2018. Disponível em:
\texttt{https://ctan.org/pkg/biblatex-abnt}. Acesso em: 6 maio 2025.

\end{document}
