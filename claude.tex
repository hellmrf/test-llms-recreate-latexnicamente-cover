% Gerado por Claude 4.5 Sonnet Thinking
\documentclass[aspectratio=169]{beamer}
\usepackage[utf8]{inputenc}
\usepackage[T1]{fontenc}
\usepackage[brazil]{babel}
\usepackage{tikz}
\usepackage{hyperref}

% Configurações do tema
\usetheme{default}
\setbeamertemplate{navigation symbols}{}
\setbeamercolor{background canvas}{bg=white}
\setbeamercolor{normal text}{fg=black}

\begin{document}

\begin{frame}[plain]
    % Citação no topo em cinza claro
    \vspace*{0.5cm}
    {\footnotesize\color{gray!50}
    LAMPORT, Leslie. \LaTeX: A Document Preparation System. 2. ed. Reading:\\
    Addison-Wesley, 1994. ISBN 9780201529838.
    }

    \vspace{1.5cm}

    % Título principal
    \begin{center}
        {\Huge\bfseries Referências Bibliográficas em}

        \vspace{0.3cm}

        {\Huge\ttfamily \textbackslash cite\{ABNT\}}
    \end{center}

    \vspace{1.5cm}

    % Segunda referência
    {\footnotesize
    MARQUES, Daniel Ballester. \textbf{Bib\LaTeX-ABNT}: Estilo para Bib\LaTeX{} compatível com\\
    as normas da ABNT. [\textit{S. l.}], 2018. Disponível em:\\
    \textcolor{blue}{\url{https://ctan.org/pkg/biblatex-abnt}}. Acesso em: 6 maio 2025.
    }

    \vspace{1cm}

    % Bandeira do Brasil (canto inferior esquerdo)
    \begin{tikzpicture}[remember picture, overlay]
        \node[anchor=south west] at ([xshift=0.5cm, yshift=0.5cm]current page.south west) {
            \begin{tikzpicture}[scale=0.6]
                % Verde
                \fill[green!50!black] (0,0) rectangle (3,2);
                % Amarelo
                \fill[yellow!80!orange] (1.5,1) -- (2.8,1) -- (1.5,0.2) -- (0.2,1) -- (1.5,1.8) -- cycle;
                % Azul
                \fill[blue!80!black] (1.5,1) circle (0.4);
                % Faixa branca
                \fill[white] (1.1,0.85) .. controls (1.3,1.1) and (1.7,1.1) .. (1.9,0.85) --
                             (1.85,0.9) .. controls (1.68,1.05) and (1.32,1.05) .. (1.15,0.9) -- cycle;
            \end{tikzpicture}
        };
    \end{tikzpicture}

    % LaTeXnicamente (canto inferior direito)
    \begin{tikzpicture}[remember picture, overlay]
        \node[anchor=south east] at ([xshift=-0.5cm, yshift=0.8cm]current page.south east) {
            {\fontsize{35}{40}\selectfont \LaTeX nicamente}
        };
    \end{tikzpicture}

\end{frame}

\end{document}
