% Gerado por GPT-5 Thinking Mini
% Slide em aspecto 16:9 que reproduz a imagem fornecida.
% Compile com pdflatex (ou lualatex/xelatex se preferir fontes do sistema).
\documentclass[aspectratio=169]{beamer}
\usetheme{default}
\setbeamertemplate{navigation symbols}{}
\setbeamercolor{background canvas}{bg=white}

\usepackage[T1]{fontenc}
\usepackage[utf8]{inputenc}
\usepackage[portuguese]{babel}

\usepackage{mathpazo} % tipografia agradável (Palatino)
\usepackage{graphicx}
\usepackage{tikz}
\usepackage{url}
\usepackage{hyperref}
\hypersetup{hidelinks}

% cores
\definecolor{lightgraytext}{RGB}{180,180,180}
\definecolor{linkblue}{RGB}{64,120,235}

% comando para "LATEXnicamente" compondo o logo LaTeX
\newcommand{\LATEXnicamente}{%
  {\Large\scshape\LaTeX}\raisebox{0.7ex}{\fontsize{18}{18}\selectfont nicamente}%
}

% pequena bandeira do Brasil em TikZ (aproximação estilizada)
\newsavebox{\brazilflag}
\sbox{\brazilflag}{%
\begin{tikzpicture}[scale=0.08,baseline=-0.45ex]
  \fill[green!60!black] (0,0) rectangle (20,14);
  \fill[yellow!90!black] (10,7) ++(-7,0) -- ++(7,7) -- ++(7,0) -- ++(-7,-7) -- cycle;
  \fill[blue] (10,7) circle (3.6);
  \fill[white] (9.35,7.1) arc (200:-20:2.6 and 0.55);
  % opcional: estrela central (simples)
  \fill[white] (10,8) circle (0.25);
\end{tikzpicture}%
}

% Para melhor layout: usamos um único frame plain
\begin{document}

\begin{frame}[plain]
  % Cabeçalho pálido no topo (citação de exemplo)
  \begin{tikzpicture}[remember picture,overlay]
    \node[anchor=north west, inner sep=12pt] at (current page.north west) {%
      \parbox{0.6\paperwidth}{\color{lightgraytext}\small
      LAMPORT, Leslie. \textsc{LaTeX}: A Document Preparation System. 2. ed. Reading: Addison-Wesley, 1994. ISBN 9780201529838.}
    };
  \end{tikzpicture}

  % Conteúdo principal centralizado verticalmente
  \vspace*{12mm}
  \begin{center}
    {\Huge\bfseries Referências Bibliográficas em}\\[6mm]

    {\LARGE\ttfamily \textbackslash cite\{{\bfseries ABNT}\}}\\[12mm]

    % Entrada bibliográfica simulada (como na imagem)
    \parbox{0.88\textwidth}{\raggedright
      {\large\scshape MARQUES, Daniel Ballester.} \textbf{Bib\LaTeX-ABNT:} \textit{Estilo para Bib\LaTeX\ compatível com as normas da ABNT}. [S. l.], 2018. Disponível em:\\[2pt]
      {\ttfamily\small\url{https://ctan.org/pkg/biblatex-abnt}}. \\
      Acesso em: 6 maio 2025.
    }
  \end{center}

  % Rodapé: bandeira à esquerda e "LATEXnicamente" centralizado
  \begin{tikzpicture}[remember picture,overlay]
    % bandeira no rodapé esquerdo
    \node[anchor=south west, inner sep=16pt] at (current page.south west) {\usebox{\brazilflag}};
    % texto central inferior
    \node[anchor=south, inner sep=6pt] at (current page.south) {\LATEXnicamente};
  \end{tikzpicture}

  % URL azul monoespaçada menor (estado similar ao da imagem)
  \begin{tikzpicture}[remember picture,overlay]
    \node[anchor=west] at ($(current page.center) + (-0.44\paperwidth,-0.22\paperheight)$) {%
      \parbox{0.5\paperwidth}{\small\ttfamily\color{linkblue}https://ctan.org/pkg/biblatex-abnt}
    };
  \end{tikzpicture}

\end{frame}

\end{document}
